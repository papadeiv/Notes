\documentclass[../main.tex]{subfiles}
\begin{document}
\section{Interval diffeomorphisms}\label{sec2}
\begin{definition}[Interval diffeomorphism]\label{def1}
     Let $I \subseteq \mathbb{R}$ a compact interval and $f:\,I\,\to I$ a $\mathcal{C}^{1}$ diffeomorphism, then the following properties hold
     \begin{itemize}
          \item endpoints are mapped to endpoints;
          \item $\newprime{f}(x)\neq 0\,\,\forall x\in I$;
          \item $f$ is either orientation preserving (i.e. $\newprime{f}(x)>0\,\,\forall x\in I$) or orientation reversing ($\newprime{f}(x)<0\,\,\forall x \in I$);
          \item if $f(p)=p$ then $p$ is hyperbolic if $\newprime{f}(p)\neq1$;
          \item if all fixed points of $f$ are hyperbolic then we say that $f$ is hyperbolic.
     \end{itemize}
\end{definition}
\begin{proposition*}
     $f:\,I\,\to I$ orientation preserving $\mathcal{C}^{1}$ diffeomorphism then $\forall x \in I$ the limit sets $\omega(x), \alpha(x)$ are fixed points for $f$.
\end{proposition*}
\begin{interpretation*}{}
     The only thing that a point $x\in I$ can do under iterations of an interval diffeomorphism $f(x)$ (forward or backward in time) is to converge to a fixed point of the map.
\end{interpretation*}
\begin{observation}[]\label{obs1}
     Let $f:\,I\,\to I$ and $g:\,J\,\to J$ two $\mathcal{C}^{1}$ diffeomorphisms s.t. they only have two fixed points each then they are topologically conjugate.
\end{observation}
\begin{lemma}
     If $p$ is a hyperbolic fixed point for $f:\,I\,\to I$ orientation preserving interval diffeomorphism then $p$ is isolted, i.e. $\exists \mathcal{U} \subset I$ s.t. $\forall x\neq p$ in $\mathcal{U}$ they are not isolted themselves.
\end{lemma}
\begin{proof}
     Exercise (Hint: by contraddiction and mean value theorem).
\end{proof}
\begin{corollary}
     If $f$ is hyperbolic then $f$ has at most a finite number of fixed points.
\end{corollary}
\begin{corollary}
     Conjugacy classes of hyperbolic, orientation preserving $\mathcal{C}^{1}-$diffeomorphism are characterised (i.e. the minimum information required to identify the conjugacy) by the number of fixed points. 
\end{corollary}
\begin{proposition*}
     If $f,g$ are two orientation preserving, hyperbolic diffeomorphisms then $f,g$ are topologically conjugate $\iff$ they have the same number of fixed points.
\end{proposition*}
\begin{proof}
     Suppose $f:\,[a,b]\,\to [a,b]$ and $g:\,[\newprime{a},\newprime{b}]\,\to [\newprime{a},\newprime{b}]$, and $a=a_{1}<a_{2}<\dots<a_{s}=b$, $\newprime{a}=\newprime{a}_{1}<\newprime{a}_{2}<\dots<\newprime{a}_{s}=\newprime{b}$.     
     \begin{figure}[H]
         \centering 
         %\includegraphics[keepaspectratio, width=\textwidth]{../figures/.png}
         \caption{Two different diffeomorphisms with different fixed points.}
         \label{fig1}
     \end{figure}
     For each pair of sub-intervals $[a_{j}, a_{j+1}]$, $[\newprime{a}_{j},\newprime{a}_{j+1}]$ we can define an homeomorphism $h_{j}:\,[a_{j},a_{j+1}]\,\to [\newprime{a}_{j},\newprime{a}_{j+1}]=h|_{[a_{j},a_{j+1}]}$ which we observe is the restriction to $[a_{j},a_{j+1}]$ of a topological conjugacy between $f$ and $g$.
\end{proof}
\begin{definition}[Uniform norm]\label{def2}
     Let $f,g:\,I\,\to I$ two continuous maps then the $\mathcal{C}^{0}-$distance between the two maps is defined as
     \begin{equation}\label{eq2}
             d_{0}(f,g):=\text{sup}_{x\in I}\{|f(x)-g(x)|\}_{}
     \end{equation}
\end{definition}
Unfortunately the uniform norm is not enough to check the structural stability of the conjugacy classes as defined above and so we modify \eqref{eq2} as such:
\begin{equation}\label{eq3}
     d_{1}(f,g)=\text{sup}_{x\in I}\{|f(x)-g(x)|+|\newprime{f}(x) - \newprime{g}(x)|\}_{}
\end{equation}
\begin{remark*}
        $d_{0}$ and $d_{1}$ induce very different topologies on the space $\mathcal{C}^{1}(I)$ of $\mathcal{C}^{1}-$maps
\end{remark*}
\begin{example}[label=ex1]{}{}
        $I=[0,1]$, $f(x)=\frac{1}{2}$ and $g(x)=\frac{1}{2}+ \epsilon \sin(\frac{x}{\epsilon})$

        $d_{0}(f,g)=\epsilon$ meanwhile $d_{1}(f,g)\geq1$ $\Rightarrow$ $d_{0}\to 0\,,\quad\epsilon\to0$ whereas $d_{1}\to1\,,\quad\epsilon\to0$.
\end{example}

\end{document}
